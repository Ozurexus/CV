%----------------------------------------------------------------------------------------
%	PACKAGES AND OTHER DOCUMENT CONFIGURATIONS
%----------------------------------------------------------------------------------------

\documentclass[9pt]{developercv} % Default font size, values from 8-12pt are recommended
\usepackage[T2A,T1]{fontenc}
\usepackage[utf8]{inputenc}
\usepackage[russian,english]{babel}
%\usepackage{newtxtext}
\usepackage{tempora} % this supports Cyrillic
\usepackage{newtxmath}
\usepackage{multicol}
\setlength{\columnsep}{0mm}

%----------------------------------------------------------------------------------------
\usepackage{lipsum} 
% \usepackage{fontawesome}
\usepackage[T2A]{fontenc} 
\hypersetup{
  colorlinks=true,
  urlcolor=blue,
}

% \documentclass[12pt]{article}


\begin{document}

%----------------------------------------------------------------------------------------
%	TITLE AND CONTACT INFORMATION
%----------------------------------------------------------------------------------------

\begin{minipage}[t]{0.5\textwidth} 
	\vspace{-\baselineskip} % Required for vertically aligning minipages
	
	{ \fontsize{16}{20} \textcolor{black}{\textbf{\MakeUppercase{Пинягин Максим Ильич}}}} % First name
	
	\vspace{6pt}
	
	{\Large Разработчик $\sim$ ДевОпс Инженер} % Career or current job title
\end{minipage}
\hfill
\begin{minipage}[t]{0.2\textwidth} % 20% of the page width for the first row of icons
	\vspace{-\baselineskip} % Required for vertically aligning minipages
	
	% The first parameter is the FontAwesome icon name, the second is the box size and the third is the text
    \icon{Globe}{11}{\href{https://ozurexus.github.io}{ozurexus.github.io}}\\ 
    \icon{Phone}{11}{+7(918)050-72-68}\\
    \icon{MapMarker}{11}{\href{https://t.me/Ozurexus}{Telegram}}\\
\end{minipage}
\begin{minipage}[t]{0.27\textwidth} % 27% of the page width for the second row of icons
	\vspace{-\baselineskip} % Required for vertically aligning minipages
	
    \icon{Envelope}{11}{\href{mailto:pinyaguinm@mail.ru}{pinyaguinm@mail.ru}}\\	
    \icon{Github}{11}{\href{https://github.com/Ozurexus}{github.com/Ozurexus}}\\
    % https://gitlab.com/Ozurexus
    \icon{Gitlab}{11}{\href{https://gitlab.com/Ozurexus}{gitlab.com/Ozurexus}}\\    
    
\end{minipage}


%----------------------------------------------------------------------------------------
%	INTRODUCTION, SKILLS AND TECHNOLOGIES
%----------------------------------------------------------------------------------------

\begin{minipage}[t]{0.46\textwidth}
    \cvsect{Обо мне}
	\vspace{-6pt}

	Целеустремленный и быстро обучаемый студент с опытом выступлений перед аудиторией на русском и английском языках. Обладаю навыками администрирования Linux, работы с базами данных PostgreSQL, MongoDB, Neo4j и базовыми знаниями программирования на Python, Bash. Имею практический опыт контейнеризации приложений с использованием Docker. Ищу возможность применить свои знания на реальных задачах.
\end{minipage}
\hfill % Whitespace between
\begin{minipage}[t]{0.465\textwidth}
    \cvsect{Навыки}
    \vspace{-6pt}
    
    \begin{minipage}[t]{0.2\textwidth}
        \textbf{Языки:}
    \end{minipage}
    \hfill
    \begin{minipage}[t]{0.73\textwidth}
       Python, PostgreSQL,  MongoDB, Neo4j, Bash,   C++, C, Java,  Svelte, JavaScript/TypeScript, Racket, Haskell.
    \end{minipage}
    \vspace{4mm}
    
    \begin{minipage}[t]{0.2\textwidth}
        \textbf{Технологии:}
    \end{minipage}
    \hfill
    \begin{minipage}[t]{0.73\textwidth}
       Docker, Kubernetes, GitHub Actions CI/CD, FireBase, Cisco Packet Tracer.
    \end{minipage}
    
\end{minipage}

%----------------------------------------------------------------------------------------
%	Projects
%----------------------------------------------------------------------------------------
\cvsect{Проекты}
\begin{entrylist}
    \entry
		{1.}
		{Helper Bot}
            {\href{https://github.com/Ozurexus/Helper-Bot}{github.com/Ozurexus/Helper-Bot}}
		{Телеграм-Бот на Питоне с помощью библиотеки aiogram. Функционал: отображение расписания пар, расчет стипендии на основе среднего балла, отображение погоды в заданной локации.}
    \entry
		{2.}
		{Planifico}
            {\href{https://github.com/Ozurexus/Planifico}{github.com/Ozurexus/Planifico}}
		{Календарь и список ежедневных дел на SvelteKit и Typescript. Авторизация через Microsoft Graph API, редактируемый список дел хранится в Google FireBase Database.}

    \entry
		{3.}
		{BookHive}
            {\href{https://github.com/Ozurexus/BookHive}
		{github.com/Ozurexus/BookHive}}
		{Сайт рекомендации книг на Питоне и Реакте. Алгоритм использует машинное обучение для запоминания вкуса пользователей и выдачи персонализированных результатов.}
    \entry
		{4.}
		{FixMyEnglish}
            {\href{https://github.com/InnoSWP/B21-07FixMyEnglish}
		{github.com/InnoSWP/B21-07FixMyEnglish}}
		{Сайт для исправления ошибок в английском языке на Flutter, использует Апи на основе нейролингвистического программирования.}
    \entry
		{5.}
		{Dockerized Telegram Bot}
            {\href{https://gitlab.com/Ozurexus/snafinalproject_pikachubot}
		{gitlab.com/Ozurexus/SnaFinalProject}}
		{Докеризированный образ телеграм бота.}
\end{entrylist}

%----------------------------------------------------------------------------------------
%	EDUCATION
%----------------------------------------------------------------------------------------
\vspace{-10 pt}
\cvsect{Образование}
\begin{entrylist}
    \entry
		{09/2021 - 06/2025}
		{{\textbf{Университет Иннополис.}}}
		{Бакалавриат}
		{Факультет компьютерных и инженерных наук. \\
            Специальность - информатика и вычислительная техника. \\
            Трек - разработка программного обеспечения. \\
            Средний балл за всё время обучения - 4.42, за последний семестр - 4.8. \\
            Среди пройденных предметов: Операционные системы, Компьютерные сети, Системное и сетевое администрирование, Проектирование баз данных, Анализ и проектирование программных систем, Распределенное и сетевое программирование.}
    \entry
		{06/2022 - 07/2022}
		{Электив "Разработка Веб-приложений". {Университет Иннополис.}}
            {}
		{}
    \entry
		{06/2023 - 07/2023}
		{Электив "Введение в интеграцию и автоматизацию процесса разработки ПО" (DevOps). {Университет Иннополис.}}
		  {}
		{}
    \entry
		{09/2023 - 12/2023}
		{Электив "Введение в программирование на квантовых компьютерах". {Университет Иннополис.}}
            {}
		{}
		 
	% \entry
	% 	{x/2023 - x/2023}
	% 	{Certificate}
	% 	{School}
	% 	{\lipsum[1][2]}
	% \entry
	% 	{x/2023 - x/2023}
	% 	{Engineer}
	% 	{School}
	% 	{\lipsum[1][2]}
\end{entrylist}

%----------------------------------------------------------------------------------------
%	EXPERIENCE
%----------------------------------------------------------------------------------------
% \vspace{-10 pt}
% \cvsect{Опыт}
% \begin{entrylist}
% 	\entry
%         {x/2023 -- x/2023}
% 		{\lipsum[1][1]}
% 		{Company}
% 		{\vspace{-10pt}
%         \begin{itemize}[noitemsep,topsep=0pt,parsep=0pt,partopsep=0pt, leftmargin=-1pt]
%             \item \lipsum[1][1-2]
%             \item \lipsum[1][3-4]
%         \end{itemize} 
%         \texttt{SQL} \slashsep \texttt{Excel}}
% 	\entry
% 		{x/2023 -- x/2023}
% 		{\lipsum[1][1]}
% 		{Company}
% 		{\vspace{-10pt}
%         \begin{itemize}[noitemsep,topsep=0pt,parsep=0pt,partopsep=0pt, leftmargin=-1pt]
%             \item \lipsum[1][1-2]
%             \item \lipsum[1][3-4]
%         \end{itemize} 
%         \texttt{SQL} \slashsep \texttt{Excel}}
% 	% \entry
% 	% 	{x/2023 -- x/2023 \\\footnotesize{scholarship holder}}
% 	% 	{\lipsum[1][1]}
% 	% 	{Company}
% 	% 	{\vspace{-10pt}
%  %        \begin{itemize}[noitemsep,topsep=0pt,parsep=0pt,partopsep=0pt, leftmargin=-1pt]
%  %            \item \lipsum[1][1-2]
%  %            \item \lipsum[1][3-4]
%  %        \end{itemize} 
%  %        \texttt{SQL} \slashsep \texttt{Excel}}
% \end{entrylist}

%----------------------------------------------------------------------------------------
%	LANGUAGES
%----------------------------------------------------------------------------------------
\vspace{-10 pt}
	\cvsect{Языки}
    \vspace{-6pt}
    
    \hspace{26mm}  \textbf{Русский язык} - родной, \textbf{Английский язык} - C2. 

%----------------------------------------------------------------------------------------

\end{document}
